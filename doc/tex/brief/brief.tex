\documentclass[a4paper]{article}

%%--LANGUAGE AND ENCODING--%%
\usepackage[english]{babel}
\usepackage[utf8]{inputenc}
\usepackage{csquotes}
\usepackage[yyyymmdd]{datetime}
\renewcommand{\dateseparator}{--}

%%--SPACING AND MARGIN--%
\usepackage[margin=1.2in]{geometry}
\def\arraystretch{1.3}

%%--SANS-SERIF FONTS FOR SECTIONS--%%
\usepackage{sectsty}
\usepackage{helvet}
\allsectionsfont{\bfseries\sffamily}


%%--ADVANCE TABULARS--%%
 \usepackage{tabularx}

%PREAMBLE%
%%-DOCUMENT INFORMATION-%%

\author{	Strand, Johan \\ \texttt{johstr@student.chalmers.se} \and
			Svedberg, Pär\\ \texttt{svpar@student.chalmers.se} \and
			Åkergren, Oskar\\ \texttt{akergren@student.chalmers.se}
}
\title{\vspace{-3cm} Project Brief  \\ SolarRemote}

%----------------------------------------------------------------------%
%-----------------------------BEGIN------------------------------------%
%----------------------------------------------------------------------%


%% See https://pingpong.chalmers.se/courseId/4905/node.do?id=2170051&ts=1421061182418&u=-668175273 page 5 for what section should include %%

\begin{document}
	\maketitle
	\section*{Background and Aim} % (fold)
	\label{sec:background_and_aim}
	Parans has a unique technology where they deliver natural sunlight through optical fiber cables. As one of only two companies in the world Parans has business all over the world and right now they have two of their biggest installations in Malaysia and Los Angeles. \\

	\noindent With the help of optical lenses, sunlight is focused in to optical fiber cables and the collecting sun panel is controlled by two stepper motors. The steering is controlled by both by input from an algorithm, which uses location from gps data and current time to calculate the sun's position i degrees, and from a sun sensor with a photocell which, in case of direct sunlight, gives accurate data to steer the sun panel in to correct position. In this way the sun light is always maximally focused in to the optical fiber cable. \\

    \noindent The panel is connected to a 12V power supply and system that controls the panel is written in C. When users want to install the panel they have to connect the panel to a laptop and give commands to the panels' system to set it in the right position.
    The communication between the panel and the laptop is via an USB port and the commands are executed from a terminal. Not all customers are used to work in terminals and the use of a personal laptop also causes problems. \\

    \noindent Our aim is therefore to develop a control pad which is easy to connect to the panels USB port and has buttons executing the commands needed for the installation of the sun panel.

	
	% section background_and_aim (end)
	\section*{Solution} % (fold)
	\label{sec:solution}
	
	% section solution (end)
	\section*{Organisation} % (fold)
	\label{sec:organisation}
	
	% section organis_ation (end)
\end{document}
