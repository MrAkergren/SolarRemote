\documentclass[a4paper]{article}

%%--LANGUAGE AND ENCODING--%%
\usepackage[english]{babel}
\usepackage[utf8]{inputenc}
\usepackage{csquotes}
\usepackage[yyyymmdd]{datetime}
\renewcommand{\dateseparator}{--}

%%--SPACING AND MARGIN--%
\usepackage[margin=1.2in]{geometry}
\def\arraystretch{1.3}

%%--SANS-SERIF FONTS FOR SECTIONS--%%
\usepackage{sectsty}
\usepackage{helvet}
\allsectionsfont{\bfseries\sffamily}


%%--ADVANCE TABULARS--%%
 \usepackage{tabularx}

%PREAMBLE%
%%-DOCUMENT INFORMATION-%%

\author{    Strand, Johan \\ \texttt{johstr@student.chalmers.se} \and
            Svedberg, Pär\\ \texttt{svpar@student.chalmers.se} \and
            Åkergren, Oskar\\ \texttt{akergren@student.chalmers.se}
}
\title{\vspace{-3cm} Project Brief  \\ SolarRemote}

%----------------------------------------------------------------------%
%-----------------------------BEGIN------------------------------------%
%----------------------------------------------------------------------%


%% See https://pingpong.chalmers.se/courseId/4905/node.do?id=2170051&ts=1421061182418&u=-668175273 page 5 for what section should include %%

\begin{document}

	\maketitle

	\section*{Background and Aim} % (fold)
	\label{sec:background_and_aim}
    	Parans has a unique technology where they deliver natural sunlight through optical fiber cables. As one of only two companies in the world Parans has business all over the world and right now they have two of their biggest installations in Malaysia and Los Angeles. \\

    	\noindent Sunlight is focused into optical fiber cables via roof mounted panels with plastic lenses, and the panel is in turn controlled by two stepper motors. The steering is controlled both by input from an algorithm and from a sun sensor. The algorithm uses location data and current time to calculate the sun's position, while the sensor has a photocell which, in direct sunlight, gives accurate data to steer the panel to correct positions. In this way, the focus of the sunlight is always optimized. \\

        \noindent The panel is connected to a 12V power supply and system that controls the panel is written in C. When users want to install the panel they have to connect the panel to a laptop and give commands to the panels' system to set it in the right position. \\
        
        \noindent The communication between the panel and the laptop is via an USB port and the commands are executed from a terminal. Not all customers are used to work in terminals and the use of a personal laptop also causes problems. \\

        \noindent Our aim is therefore to develop a control pad which is easy to connect to the panels USB port and has buttons executing the commands needed for the installation of the Parans panel.
        
    % section background_and_aim (end)

    \section*{Solution} % (fold)
    \label{sec:solution}
        The main issue in this project is to develop a device that will be able to communicate with the panel. \\

        \noindent One problem that most likely will arise is the connection over a serial link and what ways there are to handle this type of communication.

        The solution to this will be to read up on this type of communication and see how others have solved the issue. We will also be able to test our method by connecting our device to a panel available to us in the company office.\\

        \noindent As we have no previous experience with the Parans panel, we do not know how to set different commands over the serial link and then receive the reply so that will be yet another problem. Because the panel and its controller is custom build by the company, is there sparse with information sources of how the panel works.

        The solution to this will be to read the documentation send to us by the company and to utilize the 'trial-and-error' approach.

        \newpage

        \noindent The likely result of this project will be a handheld device that will be able to communicate over a serial link to the company's panel. The device will be based on a so called 'System on a Chip' (SoC) running the Linux kernel and  will be able to send commands with a push of a button. The device will contain a screen that will be able to display the information requested from the panel and will be connected by a USB-cable.

        A drawback on using an SoC as the base for the system is that our device can become relatively large compared to other devices for a similar purpose. The size will be dependent on the fact that SoC are multipurpose units and contains ports that will be left unused. A better base system could be to use a touch screen device but there are few of those devices that can act as USB-host's and then the cost would be higher then the SoC system. Another problem with touch screen devices is that the programming could become harder for our type of product.

    % section solution (end)
   \section*{Organisation} % (fold)
	\label{sec:organisation}
		Members of the project are Johan Strand, Pär Svedberg and Oskar Åkergren, all third year Computer Science bachelor students at Chalmers Technical University. The group members will alternate as project managers, having the role for one week at a time. Project managing aside, the group will share duties in researching, programming and documentation, and individual tasks will be derived from the project's milestones and democratically assigned. \\

		\noindent Information need to be gathered and read about the Arduino system and its modules, in order to gain understanding of how to best program and connect the devices. Documentation of the Parans SP3 will be provided by Parans, and a solution for the SP3 and Arduino devices to communicate with each other needs to be found. \\
		
		\noindent The preliminary schedule is based on each member committing approximately 20 hours per week to the project.
		\\ \\
		\noindent\begin{tabularx}{\textwidth}{@{}rX}
			\textbf{Course week} & \textbf{Ambition}\\
			cw 1 & Establish structure of work, startup meetings with supervisors, acquire necessary equipment. \\
			cw 2 & Read documentation, develop initial code. \\
			cw 3 & Software development \\
			cw 4 & Intermediate presentation. Deadline: Show first prototype. \\
			cw 5 & Software development. \\
			cw 6 & Consider inclusion of additional features. \\
			cw 7 & Finish software development. \\
			cw 8 & Prepare final presentation. Deadline: Finish report, present final product. \\
		\end{tabularx}
	% section organis_ation (end)
\end{document}
