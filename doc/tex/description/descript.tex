\documentclass{article}

%%--LANGUAGE AND ENCODING--%%
\usepackage[swedish]{babel}
\usepackage[utf8]{inputenc}
\usepackage{csquotes}
\usepackage[yyyymmdd]{datetime}
\renewcommand{\dateseparator}{--}

%%--BIBLOPGRAPHY--%%
%\usepackage[natbib=true, style=authoryear, citestyle=authoryear, backend=biber, dateabbrev=false, urldate=iso8601,  maxcitenames=2 ]{biblatex}

%%--SPACING AND MARGIN--%
\usepackage[margin=1.8in]{geometry}
%\usepackage{setspace} (Requires preamble)

%%--HEADER/FOORER--%%
% \usepackage{fancyhdr}

%%--SANS-SERIF FONTS FOR SECTIONS--%%
\usepackage{sectsty}
\usepackage{helvet}
\allsectionsfont{\bfseries\sffamily}

%%--MATHPACKAGES--%%
%\usepackage{mathtools}
%\usepackage{amsfonts}
%\usepackage{amsmath}

%%--GRAPHICS--%%  (Requires preamble)
% \usepackage{tikz}
% \usepackage{graphicx}

%%--ADVANCE TABULARS--%%
% \usepackage{tabularx}

%PREAMBLE%
%%-SECTION NUMMBERING DEPTH-%%
%\setcounter{secnumdepth}{3} %3=Default

%%-GRAPHICS-%%
% \DeclareGraphicsExtensions{.pdf,.png,.jpg}

%%-SPACING-%
% \doublespacing
% \setlength\parindent{0pt}

%&-BIBLIOGRAPHY-%%
%Adds references library and formats it.
% To  refere to a reference in the library use \citep{}
%\addbibresource{ref.bib} \setlength{\bibitemsep}{\baselineskip} 
%Always shows the authors in bibliography as Lastname, Firstname
%\DeclareNameAlias{sortname}{last-first} 

%%-DOCUMENT INFORMATION-%%
%Header/Footer%
\author{	Strand, Johan \\ \texttt{johstr@student.chalmers.se} \and
			Svedberg, Pär\\ \texttt{svpar@student.chalmers.se} \and
			Åkergren, Oskar\\ \texttt{akergren@student.chalmers.se}
}
\title{Projektbeskrivning  \\ SolarRemote}

% \pagestyle{fancy}
% \lhead{AUTHOR}
% \rhead{\today}
% \chead{\thepage}

\begin{document}
\section*{Projektbeskrivning -- SolarRemote} % (fold)
\label{sec:projekt}

	\subsection*{Bakgrund} % (fold)
	\label{sub:bakgrund}

	% subsection bakgrund (end)

	\subsection*{Ställda krav} % (fold)
	\label{sub:stallda_krav}

	
		\noindent Bör/skall ingå:
		\begin{itemize}
			\item run u
			\item run d
			\item run l
			\item run r
			\item run stop
			\item run auto
			\item setup
			\item restart 
		\end{itemize}

	\noindent Vid närmare eftertanke så bör dessa två, setup och restart, vara två olika då man vid vissa tillfällen vill använda enbart setup.


	\noindent Skall helst ingå:
	\begin{itemize}
		\item  diod som indikerar ”SP3-power”
		\item  diod som indikerar ”remote-pwer”
		\item  ”Get time” - date (printar till display)
		\item  ”Get pos” - lon och lat (printar till display)
	\end{itemize}

	\noindent Eventuellt kan dessa två vara samma knapp så länge displayen kan printa ”lon xx.xxxx och lat yy.yyyyy” samtidigt. Annars två	 olika. \\

	\noindent Lyx: \\
	-	”Verify position” eller liknande där GPS ger referens-tid och lon/lat och som ger signal/diod/printat på displayen som OK eller liknande. \\


	\noindent Lyx-lyx: \\
	-	”Set pos” och ”Set time” som ställer in \hbox{''lon xx.xxxxx''}, \hbox{''lat yy.yyyyyy''} och \hbox{''date 2015-01-20 13:50:32''} baserat på värden från GPS:en \\

	\noindent Lite utanför men kanske värt att diskutera: \\
-	Diod eller knapp som printar ”OK” eller ”not verified” om det är så att sensorn har tagit kommando över calculated pos. Detta kan ses genom att utläsa specifika värden i den raden som loggen skriver varje sekund. Återkommer med info om det om det blir aktuellt.
	
	% subsection stallda_krav (end)
% section projekt (end)

\end{document}
