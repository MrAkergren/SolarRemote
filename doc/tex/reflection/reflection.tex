%PACKAGES%
%%--DOCUMENT CLASS--%%
\documentclass[a4paper]{article}

%%--LANGUAGE AND ENCODING--%%
\usepackage[swedish]{babel}
\usepackage[utf8]{inputenc}
\usepackage{csquotes}
\usepackage[yyyymmdd]{datetime}
\renewcommand{\dateseparator}{--}
\usepackage[margin=4cm]{geometry}

%%-DOCUMENT INFORMATION-%%
%Header/Footer%
\author{
	Strand, Johan \\ \texttt{johstr@student.chalmers.se} \and
	Svedberg, Pär\\ \texttt{svpar@student.chalmers.se} \and
	Åkergren, Oskar\\ \texttt{akergren@student.chalmers.se}
}

\title{\vspace{-3.5cm}Reflektion \\ Scrum övning}


% \pagestyle{fancy}
% \lhead{AUTHOR}
% \rhead{\today}
% \chead{\thepage}

\begin{document}
	\maketitle
	\section*{Grupp SolarRemote} % (fold)
	\label{sec:grupp_solarremote}
	Samtliga i den här gruppen har läst kursen DAT255 ’Software Engineering’, där fokuset i kursen ligger på agilt arbete. I den kursen applicerade vi Scrum vilket innebär att vi har viss erfarenhet av den här sortens arbete, men vi ser ändå att det var bra med en repetition av grundkonceptet och viktiga delar att ha med. \\

	\noindent Övningen kändes något kaosartad men förde ändå fram viktiga punkter att tänka på när ett agilt arbetssätt appliceras. Viktiga punkter så som kommunikation mellan utvecklingslaget och produktägaren, kommunikation mellan de olika  utvecklingslagen samt att kunna leverera en produkt till deadline. \\

	\noindent Specifikt upplevde vi övningen som givande för att skapa en förståelse för att ta fram en ''Definition of Done'' och att formellt tilldela arbetsuppgifter till de olika lagmedlemmarna.
	% section grupp_solarremote (end)
	

%\printbibliography		
\end{document}
