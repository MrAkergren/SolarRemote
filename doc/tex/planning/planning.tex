\documentclass[a4paper]{article}

%%--LANGUAGE AND ENCODING--%%
\usepackage[swedish]{babel}
\usepackage[utf8]{inputenc}
\usepackage{csquotes}
\usepackage[yyyymmdd]{datetime}
\renewcommand{\dateseparator}{--}

\usepackage[hidelinks]{hyperref}

%%--BIBLOPGRAPHY--%%
%\usepackage[natbib=true, style=authoryear, citestyle=authoryear, backend=biber, dateabbrev=false, urldate=iso8601,  maxcitenames=2 ]{biblatex}

%%--SPACING AND MARGIN--%
\usepackage[margin=4cm]{geometry}
%\usepackage{setspace} (Requires preamble)

%%--HEADER/FOORER--%%
% \usepackage{fancyhdr}

%%--SANS-SERIF FONTS FOR SECTIONS--%%
\usepackage{sectsty}
\usepackage{helvet}
\allsectionsfont{\bfseries\sffamily}

%%--MATHPACKAGES--%%
%\usepackage{mathtools}
%\usepackage{amsfonts}
%\usepackage{amsmath}

%%--GRAPHICS--%%  (Requires preamble)
% \usepackage{tikz}
% \usepackage{graphicx}

%%--ADVANCE TABULARS--%%
 \usepackage{tabularx}

%PREAMBLE%
%%-SECTION NUMMBERING DEPTH-%%
%\setcounter{secnumdepth}{3} %3=Default

%%-GRAPHICS-%%
% \DeclareGraphicsExtensions{.pdf,.png,.jpg}

%%-SPACING-%
% \doublespacing
% \setlength\parindent{0pt}

%&-BIBLIOGRAPHY-%%
%Adds references library and formats it.
% To  refere to a reference in the library use \citep{}
%\addbibresource{ref.bib} \setlength{\bibitemsep}{\baselineskip} 
%Always shows the authors in bibliography as Lastname, Firstname
%\DeclareNameAlias{sortname}{last-first} 

%%-DOCUMENT INFORMATION-%%
%Header/Footer%
\author{	Strand, Johan \\ \texttt{johstr@student.chalmers.se} \and
			Svedberg, Pär\\ \texttt{svpar@student.chalmers.se} \and
			Åkergren, Oskar\\ \texttt{akergren@student.chalmers.se}
}
\title{Planeringsrapport \\ SolarRemote}

\def\arraystretch{1.3}	

% \pagestyle{fancy}
% \lhead{AUTHOR}
% \rhead{\today}
% \chead{\thepage}

\begin{document}
	\maketitle
	\thispagestyle{empty}
	
	\newpage

	\tableofcontents
	\setcounter{page}{1}
	
	\newpage

	\section{Inledning} % (fold)
	\label{sec:inledning}
	
		\subsection{Bakgrund} % (fold)
		\label{sub:bakgrund}

			Parans har utvecklat en produkt som via optiska fibrer levererar naturligt solljus.
			Som ett av få bolag i världen levererar de system globalt och deras för närvarande största installationer finns i Malaysia och Los Angeles.\\

			\noindent Med hjälp av linser fokuseras solljus in i optiska fibrer och panelen styrs med hjälp av två stegmotorer. Styrningen sker på input dels från en algoritm som, baserat på position (longitud, latitud) och tid, ger en solposition i grader och dels från en solsensor med fotocell som ger data för en finstyrning av panelens positionering då solen är framme.
			Detta för att alltid maximera solljusets fokusering in i fibern.\\

			\noindent Själva panelen körs på 12V och dess systemdesign bygger på en PIC32; koden är skriven i C. Parans kommunicerar med enheten via USB-port och PUTTy
		% subsection bakgrund (end)

		\subsection{Syfte} % (fold)
		\label{sub:problem}

			Idag styrs panelen till rätt position via en terminal/puTTy, vilket är en tröskel för Parans kunder vid installation och felsökning. 
			Exempelvis har alla inte vana av att jobba i terminaler och det kan vara krångligt att konfigurera datorns USB-portar så att kommunikation kan ske med panelen. \\

			\noindent Parans vill därför utveckla en styrdosa/box med tryckknappar, lysdioder och eventuellt en display som minskar problemen för kunderna.
			Denna box kan vara i form av ett befintligt kort som t.ex. Raspberry Pi, Arduino eller liknande men skulle också kunna vara en app för Android/iOS som kan köras på en kundens mobila enhet.
		%subsection syfte (end)

		\subsection{Avgränsningar} % (fold)
		\label{sub:avgransningar}
			
			Vi ser att detta projekt kommer kunna skapas med existerande hårdvara i form av mikrokontrollerkort, telefoner eller enkortsdatorer. Detta ger att vi kommer att begränsa projektet till dessa former och inte utveckla ett eget mönsterkort.\\

			\noindent Dagens paneler kan kommunicera med externa enheter via en USB-port men saknar övriga kommunikationsmöjligheter i dagsläget. Detta gör att projektet begränsas till kommunikation via en ansluten USB-kabel och inte via någon trådlös kommunikation.

		% subsection avgransningar (end)
	% section inledning (end)

	\newpage

	\section{Metod} % (fold)
	\label{sec:metod}
		Arbetsgången i projektet kommer att ske i en så kallad agil arbetsmetod, 'Scrum' , som vi i gruppen har utbildats i en tidigare kurs och använt med framgång. I denna arbetsmetod delas arbetet upp i små steg där delar av projektet slutförs för att kunna visas upp och granskas av produktägaren. Detta möjliggör att ändringar i projektet kan implementeras smidigare än i en vattenfallsmodell där alla krav sätts innan projektet börjar. \\

		\noindent Arbetsverktyg för att stötta oss i arbetsmetoden kommer att vara \texttt{git}, i kombination med 'GitHub' och Scrum-tillbehöret 'Waffle.io'  som är nära integrerat med GitHub. \\

		\noindent Projektets olika steg kommer att vara att först införskaffa lämplig hårdvara som når upp till satta krav. Därefter utvecklas en enklare prototyp i syfte att låta utvecklarna sätta sig in i de system som panelen använder för kommunikation och utvecklingen mot den plattform som projektet kommer nyttja, se 'Val av plattform' sidan \pageref{sec:val_av_plattform}. Slutligen kommer en mer kvalificerad programutveckling ta vid för att nå fram till målen som är uppsatta av produktägaren, i de steg som är lämpliga vid rådande tillfälle.

	% section metod (end)

	\section{Tidsplan} % (fold)
	\label{sec:section_name}
		Preliminär tidsplan för vårat projekt, där vi lägger ungefär 20 timmar per person och vecka på projektet för att nå upp till den arbetsbelastning som förväntas enligt kursen. \\

		\noindent\begin{tabularx}{\textwidth}{@{}rX}
			\textbf{Läsvecka} & \textbf{Mål}\\
			lv 1 &	Upprättande av arbetsstruktur, uppstartsmöten med handledare, införskaffande av materiel. \\
			lv 2 & Inläsning av dokumentation, utveckling av initial kod. \\
			lv 3 & Programutveckling\\
			lv 4 & Halvtidsredovisning, prototyp bör kunna uppvisas\\
			lv 5 & Programutveckling\\
			lv 6 & Undersöka vilka extra funktionaliteter som kan inkluderas\\
			lv 7 & Slutgiltig programutveckling\\
			lv 8 & Förberedelser för slutredovisning \\
		\end{tabularx}
	% section section_name (end)

	\newpage

	\section{Ställda krav} % (fold)
	\label{sub:stallda_krav}

		\begin{minipage}[t]{0.5\textwidth}
		\noindent \textbf{Kommandon som ska ingå:}
			\begin{itemize}
				\item \texttt{run u}
				\item \texttt{run d}
				\item \texttt{run l}
				\item \texttt{run r}
				\item \texttt{run stop}
				\item \texttt{run auto}
				\item \texttt{setup}
				\item \texttt{restart }
			\end{itemize}
		\end{minipage}
		\begin{minipage}[t]{0.5\textwidth}
			\noindent \textbf{Funktioner som helst ska ingå:}
			\begin{itemize}
				\item  diod som indikerar ”SP3-power”
				\item  diod som indikerar ”remote-power”
				\item  ”Get time” -- date (visas på display)
				\item  ”Get pos” -- lon och lat (visas på display)
			\end{itemize}
		\end{minipage} \\

		\vspace{2mm}\noindent 'setup' och 'restart' bör vara två olika knappar då man vid vissa tillfällen endast vill använda enbart 'setup'.\\

		\noindent 'Get pos' kan vara en knapp så länge displayen kan visa ''lon xx.xxxx'' och ''lat yy.yyyyy'' samtidigt, annars två olika. \\

		\noindent \textbf{Önskvärda funktioner steg 1} \\
		-	”Verify position” eller liknande där GPS ger referenstid och lon/lat och som ger signal/diod/utskrift på displayen som OK eller liknande. \\


		\noindent \textbf{Önskvärda funktioner steg 2} \\
		-	”Set pos” och ”Set time” som ställer in \hbox{''lon xx.xxxxx''}, \hbox{''lat yy.yyyyyy''} och \hbox{''date 2015-01-20 13:50:32''} baserat på värden från GPS:en \\

		\noindent- Diod eller knapp som skriver ut ”OK” eller ”not verified” om det är så att sensorn har tagit kommando över beräknad position. Detta kan ses genom att utläsa specifika värden i den raden som loggen skriver varje sekund.
	% section stallda_krav (end)

	\newpage

	\section{Val av plattform} % (fold)
	\label{sec:val_av_plattform}

		För att konstruera den typ av fjärrkontroll som möter projektets krav måste först ett beslut tas om vilken teknisk plattform som ska användas. De alternativ som diskuterades var androidbaserade enheter, arduinosystem och enkortsdatorer, främst Raspberry Pi.\\

		\noindent\textsf{\textbf{Androidenhet}}\\
		\begin{tabularx}{\textwidth}{@{}cXcX}
			& \textbf{Fördelar} 	& & \textbf{Nackdelar} \\
			$+$ & stor skärmyta 	&	$-$ & otydligt vilka enheter som stöder 								USB-host \\
			$+$ &  etablerat OS 	& $-$ & relativt dyr \\
			$+$	&  Touchscreen ger stor valfrihet i utförande av interface				& $-$ & mer prestanda än nödvändigt
		\end{tabularx}\\

			
		\noindent\textsf{\textbf{Raspberry Pi}}\\
		\begin{tabularx}{\textwidth}{@{}cXcX}
			& \textbf{Fördelar} 	& & \textbf{Nackdelar} \\
			$+$ & stor community med tillgång till mycket information & $-$ & mer prestanda än nödvändigt \\
			$+$ & tillgänglighet till utbyggnadsmoduler & $-$ & saknar display och knappar i grundutförande \\
			$+$ & lågt pris&	$-$ & kan bli otymplig vid användande av många tilläggsmoduler \\	
		\end{tabularx}\\

		\noindent\textsf{\textbf{Arduinosystem}}\\
		\begin{tabularx}{\textwidth}{@{}cXcX}
			& \textbf{Fördelar} 	& & \textbf{Nackdelar} \\
			$+$ & stor community med tillgång till mycket information & $-$ & saknar display och knappar i grundutförande \\
			$+$ & stor tillgänglighet till utbyggnadsmoduler & $-$ & kan bli otymplig vid användande av många tilläggsmoduler \\
			$+$ & modulär uppbyggnad \\
			$+$ & mest använda systemet till små elektronikprojekt \\
			$+$ & låg energiförbrukning \\
			$+$ & open source \\
		\end{tabularx} \\

		\vspace{2mm}\noindent Efter en mindre diskussion enades projektgruppen och uppdragsgivaren om att utgå ifrån ett arduinosystem. Beslutet grundade sig på att arduino är väl lämpat för mindre elektronikprojekt, då det har tillgång till en stor mängd tilläggsmoduler och ej innehåller fler komponenter än nödvändigt. Att plattformen är open source lämnar också möjligheten öppen för att i framtiden kunna anpassa lösningen till en egen tilläggsmodul.
	% section val_av_plattform (end)

\end{document}
